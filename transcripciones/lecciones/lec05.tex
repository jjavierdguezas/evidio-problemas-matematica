Comenzamos con la entrega de notas relacionadas con la Teoría de Números (TN) y el álebra. Aquí va a haber para todos los gustos, desde notas trilladas en nuestra enseñanza media hasta notas necesarias para alumnos de alto rendimiento.

\vspace{0.5cm}

$1\hspace{-0.05cm}\cdot\hspace{-0.05cm}\text{-}$ Un \# cualquiera puede expresarse en el sistema decimal como suma de potencias de base $10$. Por ejemplo: $\overline{abc}=100a + 10b + c$, $\overline{abcd}= 1000a + 100b + 10c + d$

\vspace{0.5cm}

$2\hspace{-0.05cm}\cdot\hspace{-0.05cm}\text{-}$ Un \# lo llamamos Cuadrado Perfecto cuando su raíz cuadrada es un número entero $(0, 1, 4, 9, 16, ..., n^2)$. Vea que la diferencia de dos (CP) consecutivos es igual a la suma de las raices cuadradas de ambos \# ejemplo: $49-36=13=7+6$

\vspace{0.5cm}

$3\hspace{-0.05cm}\cdot\hspace{-0.05cm}\text{-}$ Múltiplos y Divisores de un \#. Ejemplo: Múltiplos de $6$: $0,6,12,18,...$ Divisores de $24$: $1, 2, 3, 4, 6, 8, 24$.

\vspace{0.5cm}

$4\hspace{-0.05cm}\cdot\hspace{-0.05cm}\text{-}$ Números Primos: Son los que poseen $2$ divisores $(2,3,5,7,11,13,...)$. Números Primos entre sí: Son aquellos que solo tienen al uno como divisor común. Ejemplo: $3$ y $10$, $11$ y $17$, $4$ y $9$, $5$, $8$ y $21$ ... 

\vspace{0.5cm}

$5\hspace{-0.05cm}\cdot\hspace{-0.05cm}\text{-}$ Máximo Común Divisor (\texttt{mcd}): De todos los divisores comunes de dos o varios \#, el mayor es el \texttt{mcd}. Ejemplo: $\texttt{mcd}(24,32)=8$.

Mínimo Común Múltiplo (\texttt{mcm}): De todos los múltiplos comunes de dos o varios \#, el menor es el \texttt{mcm}. Ejemplo: $\texttt{mcm}(8,10)=40$

¿Cómo determinar el (\texttt{mcd}) y el (\texttt{mcm}) entre varios números?

Ejemplo: Sean $A=30600$, $B=4340$, $C=2674200$

\hspace{1.6cm}o sea $A=2^3\cdot5^2\cdot7\cdot19$, $B=2^2\cdot5\cdot7\cdot31$, $C=2^4\cdot5^3\cdot19^2\cdot37$

$\Longrightarrow \texttt{mcd}(A,B,C)=2^2\cdot5=20$ (Vea que se toman solo las bases comunes de las potencias elevadas al menor exponente)

\hspace{.8cm}$\texttt{mcd}(B,C)=2^2\cdot5=20$

\hspace{.8cm}$\texttt{mcd}(A,C)=2^3\cdot5^2\cdot19=3800$

\hspace{.8cm}$\texttt{mcm}(A,B,C)=2^4\cdot5^3\cdot7\cdot19^2\cdot31\cdot37$ (Vean que se toman todas las bases de las potencias y entre las comunes la de mayor exponente)

\hspace{.8cm}$\texttt{mcm}(A,B)=2^3\cdot5^2\cdot7\cdot19\cdot31$

\vspace{0.5cm}

$6\hspace{-0.05cm}\cdot\hspace{-0.05cm}\text{-}$ Reglas de divisibilidad: Un \# es divisible por ...
\begin{itemize}
    \addtolength{\itemindent}{1cm}
    \item[$2 -$] si su última cifra es par o es $= 0$
    \item[$5 -$] si su última cifra es $0$ o $5$
    \item[$10 -$] si su última cifra es $= 0$
    \item[$3 -$] si al sumar todas sus cifras se obtiene un múltiplo de $3$
    \item[$9 -$] si al sumar todas sus cifras se obtiene un múltiplo de $9$
    \item[$4 -$] si las dos últimas cifras del \# conforman un múltiplo de $4$
    \item[$8 -$] si las tres últimas cifras del \# conforman un múltiplo de $8$
    \item[$11 -$] cuando al restar los dos resultados de sumar las cifras de orden par y las de orden impar, se obtiene un múltiplo de 11, veamos... \vspace{-0.3cm}
    \item[] $\overline{abcde}$\hspace{.65cm}$S_1=b+d$\hspace{.65cm}$S_2=a+c+d$\hspace{.65cm}$|S_1 - S_2| = 11k$
    \item[$6 -$] cuando es divisible por $2$ y $3$ a la misma vez
    \item[$12 -$] cuando es divisible por $3$ y $4$ a la misma vez
    \item[$15 -$] cuando es divisible por $3$ y $5$ a la misma vez
    \item[$36 -$] cuando es divisible por $4$ y $9$ a la misma vez
    \item[] Vea que $2$ y $3$, $3$ y $4$, $3$ y $5$, $4$ y $9$ son \# primos entre sí\vspace{-0.3cm}
    \item[] Ejemplo: $1536$ es divisible por: $1,2,3,4,6,8,12,...,1536$ \vspace{-0.3cm}
    \item[]\hspace{2.5cm}no es divisible por: $5,10,9,11,36,...$
    \item[$7 -$] cuando separamos su última cifra y multiplicándola por $2$, dicho resultado se le resta al \# que resultó de suprimir la última cifra al \# original, obteniendo un múltiplo de $7$\vspace{-0.3cm}
    \item[] Ejemplo: $51492$ : $2\cdot2=4$, $5149-4=5145$, $5\cdot2=10$, $514-10=504$, $4\cdot2=8$\vspace{-0.3cm}
    \item[] $50-8=42$ $\Longrightarrow 51492$ es múltiplo de $7$
\end{itemize}

Ahora existe una regla general que nos permite obtener la regla de divisibilidad de cualquier \# primo. En cualquier caso debemos separar la última cifra del \# y multiplicarlo por un \#n $n$ y luego realizar el mismo algoritmo que la regla del $7$. El problema es ver quién es $n$.

Debemos buscar qué dígitos multiplicados por el \# al cual le estamos investigando la regla de divisibilidad , da como resultado un \# cuya última cifra es $= 1$. Del resultado de este producto nos interesa solo las cifras que queden a la izquierda del $1$, y este será el valor de $n$.

Ejemplo Regla del $97$: $97\cdot3=291 \Longrightarrow n=29$
% \vspace{-0.3cm}

\hspace{1cm}Sea $12804$: $4\cdot29=116$, $1280-116=1164$, $4\cdot29=116$, $116-116=0$

$\Longrightarrow 12804$ es divisible por $97$
\begin{center}
    \texttt{$-\cdot-\cdot-$}
\end{center}
$7\hspace{-0.05cm}\cdot\hspace{-0.05cm}\text{-}$ Divisibilidad: El número natural $n$ es un divisor del número natural $m$ si existe $x \in \mathbb{N}$ tal que $m = nx$ y se escribe $n \mid m$ o $m$ $\vdots$ $n$. Se lee $n$ divide a $m$ o $m$ es múltiplo de $n$
\begin{itemize}
    \def\labelitemi{-}
    \addtolength{\itemindent}{1cm}
    \item Para todo $n \in \mathbb{Z}_+$ ; $1 \mid n$ y $n \mid n$
    \item Si $a \mid b$ y $a \mid c \Longrightarrow a \mid b \pm c$
    \item Si $a \mid b \Longrightarrow a \mid bc$ ; $\forall c \in \mathbb{Z}$
    \item Si $a \mid b$ y $b \mid c \Longrightarrow a \mid c$
    \item Si $a \mid b$ y $a \mid c \Longrightarrow a \mid (bx + cy)$ $\forall x,y \in \mathbb{Z}$ 
    \item Si $a \mid b$ y $b \mid a \Longrightarrow a = \pm b$ 
    \item Si $a \mid b$, $a > 0$ , $b > 0 \Longrightarrow b \ge a$
\end{itemize}

\vspace{0.5cm}

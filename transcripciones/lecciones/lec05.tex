Comenzamos con la entrega de notas relacionadas con la Teoría de Números (TN) y el álebra. Aquí va a haber para todos los gustos, desde notas trilladas en nuestra enseñanza media hasta notas necesarias para alumnos de alto rendimiento.

\vspace{0.5cm}

$1\hspace{-0.05cm}\cdot\hspace{-0.05cm}\text{-}$ Un \# cualquiera puede expresarse en el sistema decimal como suma de potencias de base $10$. Por ejemplo: $\overline{abc}=100a + 10b + c$, $\overline{abcd}= 1000a + 100b + 10c + d$

\vspace{0.5cm}

$2\hspace{-0.05cm}\cdot\hspace{-0.05cm}\text{-}$ Un \# lo llamamos Cuadrado Perfecto cuando su raíz cuadrada es un número entero $(0, 1, 4, 9, 16, ..., n^2)$. Vea que la diferencia de dos (CP) consecutivos es igual a la suma de las raices cuadradas de ambos \# ejemplo: $49-36=13=7+6$

\vspace{0.5cm}

$3\hspace{-0.05cm}\cdot\hspace{-0.05cm}\text{-}$ Múltiplos y Divisores de un \#. Ejemplo: Múltiplos de $6$: $0,6,12,18,...$ Divisores de $24$: $1, 2, 3, 4, 6, 8, 24$.

\vspace{0.5cm}

$4\hspace{-0.05cm}\cdot\hspace{-0.05cm}\text{-}$ Números Primos: Son los que poseen $2$ divisores $(2,3,5,7,11,13,...)$. Números Primos entre sí: Son aquellos que solo tienen al uno como divisor común. Ejemplo: $3$ y $10$, $11$ y $17$, $4$ y $9$, $5$, $8$ y $21$ ... 

\vspace{0.5cm}

$5\hspace{-0.05cm}\cdot\hspace{-0.05cm}\text{-}$ Máximo Común Divisor (\texttt{mcd}): De todos los divisores comunes de dos o varios \#, el mayor es el \texttt{mcd}. Ejemplo: $\texttt{mcd}(24,32)=8$.

Mínimo Común Múltiplo (\texttt{mcm}): De todos los múltiplos comunes de dos o varios \#, el menor es el \texttt{mcm}. Ejemplo: $\texttt{mcm}(8,10)=40$

¿Cómo determinar el (\texttt{mcd}) y el (\texttt{mcm}) entre varios números?

Ejemplo: Sean $A=30600$, $B=4340$, $C=2674200$

\hspace{1.6cm}o sea $A=2^3\cdot5^2\cdot7\cdot19$, $B=2^2\cdot5\cdot7\cdot31$, $C=2^4\cdot5^3\cdot19^2\cdot37$

$\Longrightarrow \texttt{mcd}(A,B,C)=2^2\cdot5=20$ (Vea que se toman solo las bases comunes de las potencias elevadas al menor exponente)

\hspace{.8cm}$\texttt{mcd}(B,C)=2^2\cdot5=20$

\hspace{.8cm}$\texttt{mcd}(A,C)=2^3\cdot5^2\cdot19=3800$

\hspace{.8cm}$\texttt{mcm}(A,B,C)=2^4\cdot5^3\cdot7\cdot19^2\cdot31\cdot37$ (Vean que se toman todas las bases de las potencias y entre las comunes la de mayor exponente)

\hspace{.8cm}$\texttt{mcm}(A,B)=2^3\cdot5^2\cdot7\cdot19\cdot31$

\vspace{0.5cm}

$6\hspace{-0.05cm}\cdot\hspace{-0.05cm}\text{-}$ Reglas de divisibilidad: Un \# es divisible por ...

\hspace{.65cm}$2 -$ si su última cifra es par o es $= 0$

\hspace{.65cm}$5 -$ si su última cifra es $0$ o $5$

\hspace{.65cm}$10 -$ si su última cifra es $= 0$

\hspace{.65cm}$3 -$ si al sumar todas sus cifras se obtiene un múltiplo de $3$

\hspace{.65cm}$9 -$ si al sumar todas sus cifras se obtiene un múltiplo de $9$

\hspace{.65cm}$4 -$ si las dos últimas cifras del \# conforman un múltiplo de $4$

\hspace{.65cm}$8 -$ si las tres últimas cifras del \# conforman un múltiplo de $8$

\vspace{0.5cm}
